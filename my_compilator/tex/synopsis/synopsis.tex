\documentclass[a4paper,12pt]{article}
\usepackage[margin=24mm]{geometry}
\usepackage[utf8]{inputenc}
\usepackage[T1]{fontenc}
\usepackage[french]{babel}

\title{Linguistics term paper at University of Bergen}
\author{Koenraad De Smedt} 
\date{\today}
\title{My\_Compilator}
\begin{document}
\begin{center} \vfill
\textbf{\Large EPITA}

{\large Projet Informatique Pratique : My\_Compilator} 

\vspace{2cm}


Membres:\\

	Lucien Boillod\\
    Thibaud Michaud\\
    Maxime Gaudron\\
    Baal, Belzebuth, Morgoth, nous hesitons encore...

\vspace{5cm}

Résumé : Implémentation d'un compilateur en lagage C 
\end{center} \clearpage

\section{Etats de l'art}
Il est très difficile de faire un état de l'art eds compilateur. C'est une
discipline aussi vaste qu'indispensable dans le domain de l'informatique.
L'etats de l'art du compilateur est a l'informaticien ce que l'etats de l'art
strategique est au plus fin tacticien. Mais actuellement les compilateurs sont
de plus en plus performant et rapide. Pour en citer quelques un le compilateur
java accompagné de son ramasse-miettes et de sa machine virtuelle. Le
compilateur ocaml qui comporte lui aussi d'un ramasse-miettes. Les compilateurs
pour langage fonctionnel et/ou impératif. Bref il existe autant de compilateur
qu'il peut exister de langage (on pourrait imaginer un compilateur de francais
ou d'anglais qui comprendrais ce qu'on lui dit avec de veritable émotion et
... et ... )

\section{Présentation} \label{sec:intro}
my\_compilator consistera à implémenté un compilateur pour le langage /*inserer langage*/. 
Nous savons qu'il s'agit d'un projet ambitieux mais algorithemiquement 
interessant voir essentielle dans la vie de toute ingénieur dans les nouvelles
techonoliges se doit de connaitre et de maitriser.
Notre objectif se déroulera en plusieur partie tout d'abord la collecte
d'information, nous irons piocher dans les ressources que nous propose EPITA ,
enseignant / chercheur et bien d'autre encore. Mais aussi dans nos capassité de
recherche personnel, let me google it for you ... ainsi que nos capacité de
comprehension personnel . 
\subsection{analyse}
En un second temps nous commenceront l'implementation du compilateur voir d'un
interpreteur pour le langage choisis si cela s'y prete, la base etant semblable
au deux seul la comprehension de l'arbre syntaxique diffère.

cette etape s'articulera autours de trois points : 
\begin{itemize}
\item[->] \textbf{analyse lexical} : recuperation des "mots" permettant la comprehension du fichier
source en leur associant une signification abstraite\\
plusieur choix : lex et yacc ou from scratch
\item[->] \textbf{analyse syntaxique} : verification de la structuture
gramaticale utilisation d'un arbre syntaxique\\
\item[->] \textbf{analyse semantique} : verification du sens de la phrase
reconstruite extraction des données d'un arbre syntaxique\\
\end{itemize}

\subsection{génération de code}
Deux choix s'offre a nous a ce moment : interpretation ou compilation. Nous
nous orienterons d'abord vers la compilation du langage qui consistera aprés
l'obtentions du resultat de notre analyse sementique a faire quelques
optimisation du code code a
\end{document}
